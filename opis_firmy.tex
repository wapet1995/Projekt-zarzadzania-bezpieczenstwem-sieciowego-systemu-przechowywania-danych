\newpage\section{Opis zabezpieczanej firmy}
Rozdział zawiera charakterystykę firmy, rodzaj prowadzonej działalności, plan budynku oraz spis sprzętu i pracowników. Jest to stan biura sprzed zabezpieczenia.

\subsection{Charakterystyka firmy}
Firma jest biurem rachunkowym specjalizującym się w doradztwie \linebreak finansowym, prowadzaniu księgowości dla przedsiębiorstw oraz przygotowywaniu analizy finansowej rynku. Przedsiębiorstwo zatrudnia 39 osób, które tworzą cztery działy: dział ekonomistów, dział sprzedaży, dział IT i dział obsługi.

\subsection{Opis budynku}
Dwupiętrowy budynek firmy zlokalizowany jest na obrzeżach dużego miasta. W okolicy nie istnieje ryzyko wystąpienia klęsk żywiołowych. \linebreak Budynek otaczają stare drzewa, których nie można wyciąć, ponieważ \linebreak objęte są ochroną gatunkową. Do przedsiębiorstwa doprowadzona jest sieć \linebreak telefoniczna oraz internetowa.

\newpage
\subsection{Sprzęt oraz oprogramowanie}
Sprzęt informatyczny:

\begin{minipage}[\right]{15cm}
\begin{itemize*}
	\item urządzenie wielofunkcyjne Canon PIXMA G3400 (12 sztuk)
	\item niszczarka ProfiOffice PIRANHA EC 7 CC (12 sztuk)
	\item komputer stacjonarny (21 sztuk)
	\begin{itemize*}
		\item procesor Intel i5
		\item pamięć 8 GB RAM 
		\item dysk 1 TB HDD 
	\end{itemize*}
	\item telefon VoIP Cisco CP-7940G (21 sztuk)
	\item laptop DELL Inspiron 5567 (6 sztuk)
	\item serwer główny (1 sztuka)
	\begin{itemize*}
		\item płyta główna: Intel S2600CP4
		\item procesor Intel Xeon e5-2603 v2
		\item pamięć 128 GB RAM DDR3
		\item dyski SSD o łącznej pojemności 40 TB 
	\end{itemize*}
	\item serwer zapasowy (1 sztuka)
	\begin{itemize*}
		\item płyta główna: Intel S2600CP4
		\item procesor Intel Xeon e5-2603 v2
		\item pamięć 16 GB RAM DDR3
		\item dyski SSD o łącznej pojemności 10 TB 
	\end{itemize*}
	\item router Cisco RV325 (1 sztuka)
	\item switch główny Cisco SG300-52 (1 sztuka)
	\item bramka VoIP Grandstream HT704 (1 sztuka)
	\item switch niezarządzalny Cisco SB SF100D-16EU (7 sztuk)
	\item punkt dostępowy Asus RP-AC87 (7 sztuk)
	\item okablowanie 
	\begin{itemize*}
		\item między serwerami 1 Gb/s
		\item w pozostałych połączeniach skrętka 100 Mb/s
	\end{itemize*}
	\item UPS VOLT Micro 1200 (1 sztuka)
	\item monitoring
	\begin{itemize*}
		\item rejestrator BCS-P-QDVR0801ME z dyskiem 2 TB HDD (1 sztuka)
		\item kamera LV-IP2301IP (5 sztuk)
	\end{itemize*}
	\item taśmy magnetyczne
\end{itemize*}
\end{minipage}

Oprogramowanie
\begin{itemize}
	\item komputery pracowników działu ekonomistów
	\begin{itemize}
		\item Windows 10 ( sztuk)
		\item pakiet Office 2016 ( sztuk)
		\item pakiet Insert GT ( sztuk)
		\item Windows Defender ( sztuk)
	\end{itemize}
	\item komputery sekretariatu i działu sprzedaży
	\begin{itemize}
		\item Windows 10 ( sztuk)
		\item pakiet Office 2016 ( sztuk)
		\item Windows Defender ( sztuk)
	\end{itemize}
	\item oprogramowanie serwera i wykorzystywane technologie
	\begin{itemize}
		\item Linux Ubuntu 16.04 LTS z OpenStack (umożliwia wirtualizację
		dowolnego systemu)
		\item bazy danych MSSQL
		\item bazy danych MySQL
		\item OpenVPN
		\item Windows Server 2016 (5 sztuk)
		\item Linux Debian 8
		\item Pakiet Insert GT ( sztuk)
		\item system pocztowy Exim i Dovecot
		\begin{itemize}
			\item Roundcube jako klient poczty w przeglądarce
		\end{itemize}
		\item serwer zapasowy
		\begin{itemize}
			\item Linux Ubuntu 16.04 LTS
		\end{itemize} 
	\end{itemize}
\end{itemize}

\newpage\section{Identyfikacja zagrożeń i analiza \newline ryzyka}
W niniejszym rozdziale zostanie przeprowadzony audyt bezpieczeństwa. Zostaną przedstawione potencjalne zagrożenia m systemie.

Jednym z zagrożeń jest możliwość upadku drzewa na budynek firmy, co może spowodować pożar lub utratę prądu. W przypadku pożaru istnieje duże ryzyko utraty danych, ponieważ żadne z pomieszczeń nie posiada systemu przeciwpożarowego.  Prawdopodobieństwo wystąpienia upadku drzewa na budynek obecnie jest stosunkowo niskie. Natomiast, trzeba wziąć pod uwagę zmieniający się klimat w Polsce, który w przyszłości będzie sprzyjał powstawaniu silnych wiatrów, a tym samym prawdopodobieństwo wystąpienia tego zjawiska będzie coraz większe.
\newline Następnym zagrożeniem jest włamanie się do budynku. Zadanie nie jest trudne, ponieważ w oknach nie są stosowane alarmy, zamki na klucz czy też kraty, które utrudniłyby dostanie się do budynku. Również, drzwi nie są specjalnie zabezpieczone, a więc włamywacz przy pomocy, np. wytrychu jest w stanie w łatwy sposób dostać się do każdego pomieszczenia. Prawdopodobieństwo wystąpienia fizycznego włamania do budynku jest na średnim poziomie.(lub średnio-wysokim?). Skutki mogą być poważne. Włamywacz nie tylko może ukraść sprzęt/dane, ale także może zainstalować oprogramowanie szpiegujące.
\newline W komputerach pracowników używany jest Windows Defender, który nie jest tak skuteczny przeciwko wirusom jak produkty konkurencji. W przypadku gdy, użytkownik pobierze zainfekowany plik, istnieje średnie prawdopodobieństwo, że zawirusowany zostanie komputer czy też inne urządzenia podłączone do sieci.
\newline Hackerzy mają ułatwione zadanie związane z dostaniem się na serwery dlatego, że w serwerach nie jest używane dodatkowe oprogramowanie związane z bezpieczeństwem. Nie ma też sprzętowej zapory ogniowej, która filtrowałaby cały ruch sieciowy.	

