% !TeX spellcheck = pl_PL
\newpage\section{Identyfikacja zagrożeń \newline i analiza ryzyka}
W niniejszym rozdziale zostanie przeprowadzony audyt bezpieczeństwa. Zostaną przedstawione potencjalne zagrożenia w systemie oraz zdefiniowana zostanie metoda oceny ryzyka.

\subsection{Ocena ryzyka - metoda jakościowa}
Do oceny ryzyka wykorzystano metodę jakościową OWASP Risk Rating Methodology. W zależności od wpływu oraz prawdopodobieństwa wystąpienia zagrożenia, określa się jakie ze sobą niesie ryzyko. 
\begin{table}[!ht]
	\centering
	\caption{Kryteria oceny jakościowej}
	\label{ocenaRyzyka}
	\begin{tabular}{|c|c|c|c|c|}
		\hline 
		\multicolumn{5}{|c|}{Ryzyko}        \\ \hline
		\multirow{4}{*}{} & Wysoki  & \cellcolor{orange} Średnie     &\cellcolor{red}  Wysokie   & \cellcolor{pink} Krytyczne     \\ \cline{2-5} 
		Wpływ	  & Średni  & \cellcolor{yellow} Niskie    	 & \cellcolor{orange} Średnie    & \cellcolor{red}  Wysokie   \\ \cline{2-5}
		& Niski   & \cellcolor{green} Bardzo niskie &  \cellcolor{yellow} Niskie     & \cellcolor{orange} Średnie    \\ \cline{2-5}
		&     	& Niskie   			& Średnie    &  Wysokie    \\ \hline
		& \multicolumn{4}{c|}{Prawdopodopieństwo}  \\  \hline 
	\end{tabular}
\end{table}

\subsection{Zagrożone zasoby}
Każdy zasób należy chronić, ale nie wszystkie wymagają zabezpieczenia na jednolitym poziomie. W tabeli \ref{tab:spis_zasobow} znajduje się spis sprzętu oraz danych wraz z ich priorytetem  ważności (1 --- najważniejszy, 10 --- najniższy), które zostaną brane pod uwagę podczas przeprowadzania audytu. 

\definecolor{lightgray}{gray}{0.9}
\begin{longtable}[!ht]{|p{0.5cm}|m{4cm}|m{2cm}|m{6cm}|}
	\caption{Wykaz zasobów uwzględnianych w audycie}
	\label{tab:spis_zasobow}\\
	\hline	
	\rowcolor{lightgray}\textbf{Lp.} & \textbf{Zasób} & \textbf{Priorytet ważności} & \textbf{Opis} \\ \hline
	\rowcolor{lightgray}\multicolumn{4}{|c|}{Serwery:} \\ \hline
	1 & Kopie zapasowe & 2 & Dane potrzebne do odzyskania \linebreak sprawności systemów \\ \hline
	2 & Dyski twarde & 1 & Pamięć trwała serwera \\ \hline
	3 & Baza danych & 3 & Przechowywanie wrażliwych danych klientów \\ \hline
	4 & Zasilanie & 8 & Utrzymanie pracy serwerów \\ \hline
	\rowcolor{lightgray}\multicolumn{4}{|c|}{Komputery pracowników:} \\ \hline	
	5 & Dyski twarde & 6 & Pamięć trwała komputera \\ \hline
	6 & Kopie zapasowe & 7 & Dane potrzebne do odtworzenia \linebreak systemu \\ \hline
	7 & Dane klientów & 4 & Dane osobowe oraz finansowe klientów \\ \hline
	8 & Zasilanie komputerów stacjonarnych & 9 & Utrzymanie pracy sprzętu \\ \hline
	9 & Zasilanie laptopów & 10 &  Utrzymanie pracy sprzętu \\ \hline
	10 & Hasła użytkowników & 5 & Hasła systemowe użytkownikóa \\ \hline
	\rowcolor{lightgray}\multicolumn{4}{|c|}{Archiwum:} \\ \hline
	11 & Dokumenty papierowe & 4 & Dokumenty archiwalne klientów \\	\hline
	12 & Taśmy magnetyczne z kopiami zapasowymi & 3 & Dane potrzebne do odzyskania \linebreak  sprawności systemu \\ \hline
	\rowcolor{lightgray}\multicolumn{4}{|c|}{Inne:} \\ \hline
	13 & Rejestrator kamer & 8 & Nagrania z monitoringu \\ \hline
	14 & Sieć bezprzewodowa & 6 & Sieć połączona z siecią firmy \\ \hline
	15 & Switche sieciowe & 6 & Pośredniczą w przesyle danych przez sieć \\ \hline
	16 & Router & 8 & Umożliwia dostęp do Internetu \\ \hline
	17 & Telefony IP & 6 & Umożliwiają komunikację wewnątrz budynku \\ \hline
	18 & Bramka VoIP & 5 & Pośredniczy pomiędzy połączeniem telefonów IP \\ \hline
	19 & Zasilanie budynku & 10 & Zasilanie oświetlania, drzwi przesuwnych itp. \\ \hline
	20 & Dostępność \linebreak do Internetu & 10 & Dostęp do Internetu od dostawcy \\ \hline 
\end{longtable}

\subsection{Zagrożenia systemu}
W tym podrozdziale opisane zostaną potencjalne zagrożenia oraz w tabelach zostanie ocenione ryzyko jakie niosą ze sobą dane niebezpieczeństwa. Zagrożenia podzielono na trzy kategorie: zagrożenia naturalne, zagrożenia ludzkie oraz zagrożenia techniczne.

\subsubsection{Zagrożenia naturalne} 
Zagrożenia naturalne związane są z lokalizacją przedsiębiorstwa, należą do nich:
\begin{itemize*}
	\item zanik prądu,
	\item upadek drzewa,
	\item pożar.
\end{itemize*}

Wymienione zagrożenia mają wpływ na dostępność danych. W budynku nie ma systemu przeciwpożarowego, a więc podczas pożaru, wysoka temperatura może uszkodzić sprzęt. Również uszkodzenie sprzętu może nastąpić w czasie zaniku prądu. Zasilacze awaryjne (UPS) podczas braku prądu dostarczają energię elektryczną tylko do serwerów i pozostały sprzęt jest narażony na uszkodzenie. W tabeli 2. oceniono ryzyko związane z zagrożeniami naturalnymi.
\begin{table}[!ht]
	\centering
	\caption{Zagrożenia naturalne}
	\label{zagrożeniaNaturalne}
	\begin{tabular}{|c|c|c|c|}
		\hline
		\textbf{Podatność} & \textbf{Prawdopodobieństwo} & \textbf{Wpływ} & \textbf{Ryzyko} \\ \hline
		Zanik prądu        & Niskie                      & Średni         & Niskie          \\ \hline
		Upadek drzewa      & Niskie                      & Średni         & Niskie          \\ \hline
		Pożar              & Niskie                      & Wysoki        & Średnie         \\ \hline
	\end{tabular}
\end{table}

\subsection{Zagrożenia techniczne}
System firmy narażony jest również na zagrożenia stricte związane z informatyka (aspektem technicznym). Niebezpieczeństwa ze strony technicznej wymieniono zostały w tabeli  \ref{tab:zagrozenia_techniczne}. Określono stopień ryzyka dla każdego istotnego zasobu systemu.

\definecolor{verylightgray}{gray}{0.5}
\begin{landscape}
\begin{longtable}[!ht]{|m{4cm}|m{6cm}|m{4.5cm}|m{3cm}|m{3cm}|}
	\caption{Wykaz zagrożeń technicznych}
	\label{tab:zagrozenia_techniczne}\\
	\hline	
	\textbf{Podatność} & \textbf{Zasoby} & \textbf{Prawdopodobieństwo} & \textbf{Wpływ} &  \textbf{Ryzyko} \\ \hline
	\multirow{5}{4cm}{Złamanie hasła administratora dowolnego komputera}  
		&   Serwer --- baza danych,  & Wysokie & Wysoki & \textcolor{pink}{Krytyczne}  \\ \cline{2-5}
		& Komputery pracowników --- dane klientów & Wysokie & Wysoki & \textcolor{pink}{Krytyczne} \\ \cline{2-5}
		& Komputery pracowników --- hasła użytkowników & Średnie & Wysoki & \textcolor{red}{Wysokie} \\ \cline{2-5}
		& Komputery pracowników --- kopie zapasowe & Wysokie & Średni & \textcolor{red}{Wysokie} \\ \cline{2-5}
		& Bramka VoIP & Średnie & Średni & \textcolor{orange}{Średnie} \\ \cline{2-5}
		& Telefon IP & Niskie & Średni & \textcolor{yellow}{Niskie} \\ \cline{2-5}
		& Rejestrator kamer & Niskie & Średni & \textcolor{yellow}{Niskie} \\ \cline{2-5}
		& Pozostałe zasoby & Niski & Niski & \textcolor{green}{Bardzo niskie} \\ \cline{2-5}
	\hline
	\multirow{11}{4cm}{Szkodliwe oprogramowanie} 
		& Serwer --- kopie zapasowe & Średnie & Wysoki & \textcolor{red}{Wysokie} \\ \cline{2-5}
		& Serwer --- dyski twarde & Średnie & Wysoki & \textcolor{red}{Wysokie} \\ \cline{2-5}
		& Serwer --- baza danych & Średnie & Wysoki & \textcolor{red}{Wysokie} \\ \cline{2-5}
		& Komputery pracowników --- dyski twarde & Wysokie & Średni & \textcolor{red}{Wysokie} \\ \cline{2-5}
		& Komputery pracowników --- kopie zapasowe & Średnie & Średni & \textcolor{orange}{Średnie} \\ \cline{2-5}
		& Komputery pracowników --- dane klientów & Wysokie & Wysoki & \textcolor{pink}{Krytyczne} \\ \cline{2-5}
		& Komputery pracowników --- hasła użytkowników & Średnie & Wysoki & \textcolor{red}{Wysokie} \\ \cline{2-5}
		& Sieć bezprzewodowa & Niskie & Średni & \textcolor{yellow}{Niskie} \\ \cline{2-5}
		& Bramka VoIP & Średnie & Średni & \textcolor{orange}{Średnie} \\ \cline{2-5}
		& Telefon IP & Średnie & Średni & \textcolor{orange}{Średnie} \\ \cline{2-5}
		& Pozostałe zasoby & Niskie & Niski & \textcolor{green}{Bardzo niskie} \\ \cline{2-5}
	\hline
	\multirow{8}{4cm}{Infekcja komputera wirusem typu ransomware}
		& Serwer --- kopie zapasowe & Wysokie & Wysoki & \textcolor{pink}{Krytyczne} \\ \cline{2-5}
		& Serwer --- dyski twarde & Wysokie & Wysoki & \textcolor{pink}{Krytyczne} \\ \cline{2-5}
		& Serwer --- baza danych & Wysokie & Wysoki & \textcolor{pink}{Krytyczne} \\ \cline{2-5}
		& Komputery pracowników --- dyski twarde & Wysokie & Średni & \textcolor{red}{Wysokie} \\ \cline{2-5}
		& Komputery pracowników --- kopie zapasowe & Wysokie & Średni & \textcolor{red}{Wysokie} \\ \cline{2-5}
		& Komputery pracowników --- dane klientów & Wysokie & Średni & \textcolor{red}{Wysokie} \\ \cline{2-5}
		& Rejestrator kamer & Wysokie & Niski & \textcolor{orange}{Średnie} \\ \cline{2-5}
		& Pozostałe zasoby & Niskie & Niski & \textcolor{green}{Bardzo niskie} \\ \cline{2-5}
 	\hline 
 	\multirow{6}{4cm}{Zużycie sprzętu (dysk, zasilacz, inne podzespoły)}
 		& Serwer --- dyski twarde & Wysokie & Wysoki & \textcolor{pink}{Krytyczne} \\ \cline{2-5}
 		& Serwer --- kopie zapasowe & Wysokie & Wysoki & \textcolor{pink}{Krytyczne} \\ \cline{2-5}
 		& Serwer --- dyski twarde & Wysokie & Wysoki & \textcolor{pink}{Krytyczne} \\ \cline{2-5}
 		& Serwer --- zasilanie & Wysokie &Średni & \textcolor{red}{Wysokie} \\ \cline{2-5}
 		& Komputery pracowników --- dyski twarde & Niskie & Średni & \textcolor{yellow}{Niskie} \\ \cline{2-5}
 		& Komputery pracowników --- kopie zapasowe & Niskie & Średni & \textcolor{yellow}{Niskie} \\ \cline{2-5}
 	\hline
\end{longtable}
\end{landscape}

\subsubsection{Zagrożenia ludzkie}

\begin{landscape}
\begin{longtable}[!ht]{|m{4cm}|m{6cm}|m{4.5cm}|m{3cm}|m{3cm}|}
	\caption{Wykaz zagrożeń ludzkich}
	\label{tab::zagrozenia_ludzkie} \\
		\hline
		\textbf{Podatność} & \textbf{Zasoby} & \textbf{Prawdopodobieństwo} & \textbf{Wpływ} &  \textbf{Ryzyko} \\ \hline
		\multirow{8}{4cm}{kradzież sprzętu oraz dokumentów przez pracowników lub osoby spoza firmy}             & Serwer - kopie zapasowe                           & średnie            & wysoki & wysokie       \\ \cline{2-5} 
		& Serwer - dyski twarde                             & średnie            & wysoki & wysokie       \\ \cline{2-5} 
		& Serwer - baza danych                              & średnie            & wysoki & wysokie       \\ \cline{2-5} 
		& Komputery pracowników - kopie zapasowe            & średnie            & wysoki & wysokie       \\ \cline{2-5} 
		& Komputery pracowników - dyski twarde              & średnie            & wysoki & wysokie       \\ \cline{2-5} 
		& Archiwum - dokumenty papierowe                    & średnie            & wysoki & wysokie       \\ \cline{2-5} 
		& Archiwum -taśmy magnetyczne z kopiami zapasowymi  & średnie            & wysoki & wysokie       \\ \cline{2-5} 
		& Pozostałe                                         & niskie             & średni & niskie        \\ \hline
		\multirow{7}{4cm}{zainstalowanie zainfekowanego oprogramowania przez pracowników lub osoby spoza firmy} & Komputery pracowników - dyski twarde              & wysokie            & wysoki & krytyczne     \\ \cline{2-5} 
		& Komputery pracowników - dane klientów             & wysokie            & wysoki & krytyczne     \\ \cline{2-5} 
		& Komputery pracowników - kopie zapasowe            & wysokie            & wysoki & krytyczne     \\ \cline{2-5} 
		& Rejestrator kamer                                 & średnie            & średni & średnie       \\ \cline{2-5} 
		& Router                                            & średnie            & wysoki & wysokie       \\ \cline{2-5} 
		& Dostęp do Internetu                               & średnie            & wysoki & wysokie       \\ \cline{2-5} 
		& Pozostałe                                         & niskie             & niski  & bardzo niskie \\ \hline
		\newpage
		\hline
		\multirow{8}{4cm}{zniszczenie sprzętu przez pracowników lub osoby spoza firmy}
		& Serwer - kopie zapasowe                           & niskie             & wysoki & średnie       \\ \cline{2-5} 
		& Serwer - dyski twarde                             & niskie             & wysoki & średnie       \\ \cline{2-5} 
		& Serwer - baza danych                              & niskie             & wysoki & średnie       \\ \cline{2-5} 
		& Komputery pracowników - dyski twarde              & średnie            & wysoki & wysokie       \\ \cline{2-5} 
		& Komputery pracowników - kopie zapasowe            & średnie            & wysoki & wysokie       \\ \cline{2-5} 
		& Komputery pracowników - dane klientów             & średnie            & wysoki & wysokie       \\ \cline{2-5} 
		& Archiwum - taśmy magnetyczne z kopiami zapasowymi & niskie             & wysoki & średnie       \\ \cline{2-5} 
		& Pozostałe                                         & niskie             & średni & niskie        \\ \hline
		\multirow{5}{4cm}{usunięcie danych przez pracowników lub osoby spoza firmy}                             & Serwer - kopie zapasowe                           & średnie            & wysoki & wysokie       \\ \cline{2-5} 
		& Serwer - baza danych                              & średnie            & wysoki & wysokie       \\ \cline{2-5} 
		& Komputery pracowników - kopie zapasowe            & średnie            & średni & średnie       \\ \cline{2-5} 
		& Komputery pracowników - dane pracowników          & średnie            & wysoki & wysokie       \\ \cline{2-5} 
		& Komputery pracowników - dyski twarde              & średnie            & wysoki & wysokie       \\ \hline
		\newpage
		\hline
		\multirow{6}{4cm}{nieautoryzowana zmiana treści dokumentów przez pracowników lub osoby spoza firmy}     & Serwer - kopie zapasowe                           & niskie             & średni & niskie        \\ \cline{2-5} 
		& Serwer - baza danych                              & średnie            & wysoki & wysokie       \\ \cline{2-5} 
		& Komputery pracowników - kopie zapasowe            & niskie             & średni & niskie        \\ \cline{2-5} 
		& Komputery pracowników - dane klientów             & średnie            & wysoki & wysokie       \\ \cline{2-5} 
		& Archiwum - dokumenty papierwowe                   & niskie             & średni & niskie        \\ \cline{2-5} 
		& Archiwum - taśmy magnetyczne z kopiami zapasowymi & niskie             & średni & niskie        \\ \hline
		\multirow{8}{4cm}{atak hakerski}                                                                        & Serwer - kopie zapasowe                           & wysokie            & wysoki & krytyczne     \\ \cline{2-5} 
		& Serwer - dyski twarde                             & wysokie            & wysoki & krytyczne     \\ \cline{2-5} 
		& Serwer - baza danych                              & wysokie            & wysoki & krytyczne     \\ \cline{2-5} 
		& Komputery pracowników - kopie zapasowe            & wysokie            & wysoki & krytyczne     \\ \cline{2-5} 
		& Komputery pracowników - dyski twarde              & wysokie            & wysoki & krytyczne     \\ \cline{2-5} 
		& Komputery pracowników - dane klientów             & wysokie            & wysoki & krytyczne     \\ \cline{2-5} 
		& Router                                            & wysokie            & wysoki & krytyczne     \\ \cline{2-5} 
		& Pozostałe                                         & średnie            & wysoki & wysokie       \\ \hline
\end{longtable}
\end{landscape}

To na dole jest do zrobienie :)  \\
Jednym z zagrożeń jest możliwość upadku drzewa na budynek firmy, co może spowodować pożar lub utratę prądu. W przypadku pożaru istnieje duże ryzyko utraty danych (wpływ na dostępność danych), ponieważ żadne z~pomieszczeń nie posiada systemu przeciwpożarowego.  Prawdopodobieństwo wystąpienia upadku drzewa na budynek obecnie jest stosunkowo niskie. Natomiast, trzeba wziąć pod uwagę zmieniający się klimat w Polsce, który w~przyszłości będzie sprzyjał powstawaniu silnych wiatrów, a tym samym prawdopodobieństwo wystąpienia tego zjawiska będzie coraz większe. \\ \colorbox{yellow}{Wpływ - średni, prawdopodobieństwo - niskie, ryzyko - niskie.}

Następnym zagrożeniem jest włamanie się do budynku. Zadanie nie jest trudne, ponieważ w oknach nie są stosowane alarmy, zamki na klucz czy też kraty, które utrudniłyby dostanie się do budynku. Również, drzwi nie są specjalnie zabezpieczone, a więc włamywacz przy pomocy, np. wytrychu jest w stanie w łatwy sposób dostać się do każdego pomieszczenia. Prawdopodobieństwo wystąpienia fizycznego włamania do budynku jest na średnim poziomie. Skutki mogą być poważne. Włamywacz nie tylko może ukraść sprzęt/dane (zagrożenie zasad poufności danych), ale także może zainstalować oprogramowanie szpiegujące lub zmodyfikować istniejące dane (wpływ na zasady integralności danych). \\ \colorbox{orange}{Wpływ - wysoki, prawdopodobieństwo - średnie, ryzyko - wysokie.}

W komputerach pracowników używany jest Windows Defender, który nie jest tak skuteczny przeciwko wirusom jak produkty konkurencji. W przypadku gdy, użytkownik pobierze zainfekowany plik, istnieje średnie prawdopodobieństwo, że zawirusowany zostanie komputer, czy też inne urządzenia podłączone do sieci (naruszenie zasad integralności oraz poufności). \\ \colorbox{orange}{Wpływ - wysoki, prawdopodobieństwo - średnie, ryzyko - wysokie.}

Hackerzy mają ułatwione zadanie związane z dostaniem się na serwery dlatego, że w serwerach nie jest używane dodatkowe oprogramowanie związane z bezpieczeństwem (złamanie zasady poufności oraz wysokie zagrożenie integralności danych). Nie ma też sprzętowej zapory ogniowej, która filtrowałaby cały ruch sieciowy. \\ \colorbox{pink}{Wpływ - wysoki, prawdopodobieństwo - wysokie, ryzyko - krytyczne.}

Firma nie jest odpowiednio przygotowana na spadki napięcia lub zanik prądu, ponieważ w czasie awarii zasilania wszystkie komputery, kamery \linebreak razem z rejestratorem obrazu i inne urządzenia elektryczne wyłączą się. Nagłe wyłączenie się komputerów może spowodować utratę ważnych danych lub też uszkodzenie komponentów w komputerze (wysoki stopnień naruszenia zasad dostępności oraz integralności). Wyłączenie się kamer utrudnia ochronę budynku. Zasilacze awaryjne (UPS) dostarczają energię elektryczną tylko do serwerów. \\ \colorbox{red}{Wpływ - wysoki, prawdopodobieństwo - średnie, ryzyko - wysokie.}

Brak wystarczającej ilości kamer, aby odpowiednio monitorować budynek z zewnątrz i wewnątrz. Kamery przy wejściu do budynku i w sekretariacie na parterze mają zbyt dużą powierzchnię do monitorowania, co powoduje istnienie martwych pól przez długi czas. Wówczas poruszając się poza widocznością kamery można włamać się (skutki zostały opisane wcześniej). 

Monitory obrócone w stronę okna umożliwiają podejrzenie co dana osoba wykonuje na komputerze, zatem naruszona zostaje poufności przeglądanych materiałów. \\ \colorbox{yellow}{Wpływ - średni, prawdopodobieństwo - niskie, ryzyko - niskie.}

Brak polityki bezpieczeństwa powoduje sytuację, w której użytkownicy mogą używać prostych haseł, czy też przez długi okres czasu nie zmieniać ich. Uruchomiony odpowiedni program może złamać takie hasła niezależnie od skomplikowania hasła, również hasło może być poznane w stosunkowo krótkim czasie, czy będzie trywialne. Uzyskanie dostępu do sytemu przez nieupoważnione osoby spowoduje naruszenie zasady poufności. Osoba taka może również zmienić hasło użytkownika, co ograniczy dostępność do zasobów, można również zmodyfikować dane naruszając integralność. \\  \colorbox{red}{Wpływ - wysoki, prawdopodobieństwo - wysokie, ryzyko - krytyczne.}

Niezabezpieczone porty USB umożliwiają użytkownikowi nie tylko kopiowanie danych z firmy, ale również zainfekowanie komputera, czy też jego uszkodzenie przy użyciu np. USB Killer (zagrożenie dla poufności danych, dostępności oraz integralności). \\ \colorbox{pink}{Wpływ - wysoki, prawdopodobieństwo - wysokie, ryzyko - krytyczne.}

Nieużywanie programów do blokowania instalacji programów ułatwia użytkownikowi wgranie dowolnego oprogramowania. Bez blokowania  dostępu do wybranych stron internetowych, użytkownicy mogą wchodzić na strony zainfekowane (zgrożenie dla poufności danych i integralności). \\ \colorbox{orange}{Wpływ - wysoki, prawdopodobieństwo - niskie, ryzyko - średnie.} 

Bezpieczeństwo poczty elektronicznej jest zapewnione przez używanie odpowiedniego oprogramowania. Natomiast, komunikacja w celu wysłania pliku między komputerami lub między komputerem a drukarką nie jest zabezpieczona. Istnieje więc ryzyko zmodyfikowania pliku, bądź też jego przejęcia przez nieautoryzowaną osobę  (zagrożenie dla poufności danych i integralności). \\   \colorbox{orange}{Wpływ - wysoki, prawdopodobieństwo - niskie, ryzyko - średnie.}  

Kolejnym zagrożeniem jest topologia sieci internetowej, a konkretniej połączenie poprzez sieć bezprzewodową. Istnieje możliwość wykorzystania sieci otwartej WiFi do przedostania się do innych podsieci, a nawet \linebreak w~skrajnym przypadku do serwerów (naruszenie poufności oraz integralności danych). Również niebezpieczeństwem \linebreak związanym z siecią jest brak alternatywnego połączenia z Internetem (brak zapewnienia dostępności). \\ \colorbox{red}{Wpływ - wysoki, prawdopodobieństwo - średnie, ryzyko - wysokie.}


% uwzględnić rodzaj danych: tajne, ściśle tajne, wrazliwe
% wyszczególnić wpływ zagrożenia na poufność, dostepność i itegralność
% sprecyzować prawdopodobieństwo ryzyka
% 
