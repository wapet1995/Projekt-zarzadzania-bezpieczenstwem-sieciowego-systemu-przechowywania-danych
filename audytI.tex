\newpage\section{Identyfikacja zagrożeń \newline i analiza ryzyka}
W niniejszym rozdziale zostanie przeprowadzony audyt bezpieczeństwa. Zostaną przedstawione potencjalne zagrożenia m systemie.

Jednym z zagrożeń jest możliwość upadku drzewa na budynek firmy, co może spowodować pożar lub utratę prądu. W przypadku pożaru istnieje duże ryzyko utraty danych, ponieważ żadne z pomieszczeń nie posiada systemu przeciwpożarowego.  Prawdopodobieństwo wystąpienia upadku drzewa na budynek obecnie jest stosunkowo niskie. Natomiast, trzeba wziąć pod uwagę zmieniający się klimat w Polsce, który w przyszłości będzie sprzyjał powstawaniu silnych wiatrów, a tym samym prawdopodobieństwo wystąpienia tego zjawiska będzie coraz większe.

Następnym zagrożeniem jest włamanie się do budynku. Zadanie nie jest trudne, ponieważ w oknach nie są stosowane alarmy, zamki na klucz czy też kraty, które utrudniłyby dostanie się do budynku. Również, drzwi nie są specjalnie zabezpieczone, a więc włamywacz przy pomocy, np. wytrychu jest w stanie w łatwy sposób dostać się do każdego pomieszczenia. Prawdopodobieństwo wystąpienia fizycznego włamania do budynku jest na średnim poziomie.(lub średnio-wysokim?). Skutki mogą być poważne. Włamywacz nie tylko może ukraść sprzęt/dane, ale także może zainstalować oprogramowanie szpiegujące.

W komputerach pracowników używany jest Windows Defender, który nie jest tak skuteczny przeciwko wirusom jak produkty konkurencji. W przypadku gdy, użytkownik pobierze zainfekowany plik, istnieje średnie prawdopodobieństwo, że zawirusowany zostanie komputer czy też inne urządzenia podłączone do sieci.

Hackerzy mają ułatwione zadanie związane z dostaniem się na serwery dlatego, że w serwerach nie jest używane dodatkowe oprogramowanie związane z bezpieczeństwem. Nie ma też sprzętowej zapory ogniowej, która filtrowałaby cały ruch sieciowy.	