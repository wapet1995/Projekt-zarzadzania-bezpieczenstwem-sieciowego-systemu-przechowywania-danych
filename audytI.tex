% !TeX spellcheck = pl_PL
\newpage\section{Identyfikacja zagrożeń \newline i analiza ryzyka}
W niniejszym rozdziale zostanie przeprowadzony audyt bezpieczeństwa. Zostaną przedstawione potencjalne zagrożenia w systemie.

Jednym z zagrożeń jest możliwość upadku drzewa na budynek firmy, co może spowodować pożar lub utratę prądu. W przypadku pożaru istnieje duże ryzyko utraty danych (wpływ na dostępność danych), ponieważ żadne z~pomieszczeń nie posiada systemu przeciwpożarowego.  Prawdopodobieństwo wystąpienia upadku drzewa na budynek obecnie jest stosunkowo niskie. Natomiast, trzeba wziąć pod uwagę zmieniający się klimat w Polsce, który w~przyszłości będzie sprzyjał powstawaniu silnych wiatrów, a tym samym prawdopodobieństwo wystąpienia tego zjawiska będzie coraz większe.

Następnym zagrożeniem jest włamanie się do budynku. Zadanie nie jest trudne, ponieważ w oknach nie są stosowane alarmy, zamki na klucz czy też kraty, które utrudniłyby dostanie się do budynku. Również, drzwi nie są specjalnie zabezpieczone, a więc włamywacz przy pomocy, np. wytrychu jest w stanie w łatwy sposób dostać się do każdego pomieszczenia. Prawdopodobieństwo wystąpienia fizycznego włamania do budynku jest na średnim poziomie.(lub średnio-wysokim?). Skutki mogą być poważne. Włamywacz nie tylko może ukraść sprzęt/dane (zagrożenie zasad poufności danych), ale także może zainstalować oprogramowanie szpiegujące lub zmodyfikować istniejące dane (wpływ na zasady integralności danych).

W komputerach pracowników używany jest Windows Defender, który nie jest tak skuteczny przeciwko wirusom, jak produkty konkurencji. W przypadku gdy, użytkownik pobierze zainfekowany plik, istnieje średnie prawdopodobieństwo, że zawirusowany zostanie komputer, czy też inne urządzenia podłączone do sieci (naruszenie zasad integralności oraz poufności).

Hackerzy mają ułatwione zadanie związane z dostaniem się na serwery dlatego, że w serwerach nie jest używane dodatkowe oprogramowanie związane z bezpieczeństwem (złamanie zasady poufności oraz wysokie zagrożenie integralności danych). Nie ma też sprzętowej zapory ogniowej, która filtrowałaby cały ruch sieciowy.	

Firma nie jest odpowiednio przygotowana na spadki napięcia lub zanik prądu, ponieważ w czasie awarii zasilania wszystkie komputery, kamery \linebreak razem z rejestratorem obrazu i inne urządzenia elektryczne wyłączą się. Nagłe wyłączenie się komputerów może spowodować utratę ważnych danych lub też uszkodzenie komponentów w komputerze (wysoki stopnień naruszenia zasad dostępności oraz integralności). Wyłączenie się kamer utrudnia ochronę budynku. Zasilacze awaryjne (UPS) dostarczają energię elektryczną tylko do serwerów. 

Monitory obrócone w stronę okna umożliwiają podejrzenie co dana osoba wykonuje na komputerze, zatem naruszona zostaje poufności przeglądanych materiałów.

Brak wystarczającej ilości kamer, aby odpowiednio monitorować budynek z zewnątrz i wewnątrz. Kamery przy wejściu do budynku i w sekretariacie na parterze mają zbyt dużą powierzchnię do monitorowania, co powoduje istnienie martwych pól przez długi czas. Wówczas poruszając się poza widocznością kamery można włamać się (skutki zostały opisane wcześniej).  

Brak polityki bezpieczeństwa powoduje sytuację, w której użytkownicy mogą używać prostych haseł, czy też przez długi okres czasu nie zmieniać ich. Uruchomiony odpowiedni program może złamać takie hasła niezależnie od skomplikowania hasła, również hasło może być poznane w stosunkowo krótkim czasie, czy będzie trywialne. Uzyskanie dostępu do sytemu przez nieupoważnione osoby spowoduje naruszenie zasady poufności. Osoba taka może również zmienić hasło użytkownika, co ograniczy dostępność do zasobów, można również zmodyfikować dane naruszając integralność.

Niezabezpieczone porty USB umożliwiają użytkownikowi nie tylko kopiowanie danych z firmy, ale również zainfekowanie komputera (zagrożenie dla poufności danych oraz integralności).  

Nieużywanie programów do blokowania instalacji programów ułatwia użytkownikowi wgranie dowolnego oprogramowania. Bez blokowania  dostępu do wybranych stron internetowych, użytkownicy mogą wchodzić na strony zainfekowane lub też „kopiące” kryptowaluty.

Kolejnym zagrożeniem jest topologia sieci internetowej, a konkretniej połączenie poprzez sieć bezprzewodową. Istnieje możliwość wykorzystania sieci otwartej WiFi do przedostania się do innych podsieci, a nawet \linebreak w~skrajnym przypadku do serwerów (naruszenie poufności oraz integralności danych). Również niebezpieczeństwem \linebreak związanym z siecią jest brak alternatywnego połączenia z Internetem (brak zapewnienia dostępności).

% uwzględnić rodzaj danych: tajne, ściśle tajne, wrazliwe
% wyszczególnić wpływ zagrożenia na poufność, dostepność i itegralność
% sprecyzować prawdopodobieństwo ryzyka
% 
